\chapter{Procassamento das imagens} \label{processamento}

\section{Introdução}

Para esse teste foram utlizadas duas imagens, uma de um avião e outra de um castelo
a fim de obter as bordas das imagens.

\section{Avião}

A imagem original do avião foi transformada em Greyscale. Após isso, foi obtida
as bordas dessa imagem, extraindo o contorno e sobrepondo-o sobre a imagem original.

% \begin{figure}[H]
%     \centering
%     \includegraphics[scale=1]{../img/aviao/processamento.png}
%     \caption{Imagens intermediárias utlizadas para o processamento da imagem do avião}
%     \label{aviaoProcessamento} 
% \end{figure}

\section{Castelo}

A imagem original do castelo foi transformada em Greyscale. Após isso, foi 
utilizado blur, para que fosse não fosse detectado os tijolos individuais do castelo,
 pois o objetivo era obter somente o contorno do castelo. Finalmente, foi extraido
 o contorno e sobrepondo-o sobre a imagem original.

% \begin{figure}[H]
%   \centering
%   \includegraphics[scale=1]{../img/castelo/processamento.png}
%   \caption{Imagens intermediárias utlizadas para o processamento da imagem do castelo}
%   \label{casteloProcessamento} 
% \end{figure}