% Headings
\chapter{Procassamento das imagens} \label{processamento}
\section{Introdução}

% images
\begin{figure}[H]
    \centering
    \includegraphics[scale=1]{./img/}
    \caption{Caption}
    \label{label} 
\end{figure}

% Tables

\begin{table}[H]
  \centering
  \caption{Cronograma}
  \begin{adjustbox}{width=\textwidth}
  \begin{tabular}{|c|c|c|c|c|c|c|c|c|c}
  \hline
  ID & Descrição & Dezembro & Janeiro & Fevereiro & Março & Abril & Maio & Junho\\
  \hline
  AT1 & Revisão bibliográfica & X & X & X & X & X & X & \\
  AT2 & Elaboração de modelos 3D & X & X & & & & & \\
  AT3 & Elaboração de cenário & X & X & & & & & \\
  AT4 & Implementação de mecânicas de jogo & X & X & X & & & &\\
  AT5 & Implementação de interface do jogo & X & X & X & X & & &\\
  AT6 & Implementação da simulação (PGC) & & X & X & X & & &\\
  AT7 & Correções e ajustes & & & & & X & &\\
  AT8 & \textit{PlayTest} & & & & & X & X &\\
  AT9 & Avaliar resultados & & & & & & X &\\
  AT10 & Elaboração da apresentação & & & & & & X &\\
  AT11 & Defesa do TCC & & & & & & & X\\
  \hline
  \end{tabular}
  \end{adjustbox}
  \smallcaption{Fonte: Autores} 
  \label{table:cronograma}
\end{table}

% List

\begin{itemize}
  \item Blender (Modelagem 3D)
  \item Paint.net (Edição de imagem simples)
  \item Gimp (Edição de imagem avançada)
  \item Krita (Software para desenho digital)
  \item Audacity (Edição de áudio)
\end{itemize}

\begin{itemize}[itemindent=!]
  \item \texttt{ctrl.Rule(eat['pouco'], weight['leve'])}
  \item \texttt{ctrl.Rule(eat['razoavel'], weight['medio'])}
  \item \texttt{ctrl.Rule(eat['muito'], weight['pesado']))}
\end{itemize}